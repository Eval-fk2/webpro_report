\chapter{Orangestar曲一覧 利用者向け仕様書}

\section{サービス概要}
Orangestar曲一覧とは,ボカロP,Orangestar氏によって作られた多くの楽曲の情報をまとめたサービスである.

\begin{figure}[H]
  \centering
  \includegraphics[width=11cm\textwidth]{fig/page_move.png}
  \caption{ページ遷移図}
  \label{fig:page_move}
\end{figure}

\section{アクセス方法}
\url{http://localhost:8080/db}\\
から,一覧表示ページにアクセスできる.

\section{一覧表示ページ}
\begin{figure}[H]
  \centering
  \includegraphics[width=15cm\textwidth]{fig/os_list_page.png}
  \caption{一覧表示ページ}
  \label{fig:os_list_page}
\end{figure}

一覧表示ページでは,登録されている楽曲の情報を一覧で見ることができる.\\
下記の情報が表示される.
\begin{itemize}
    \item id
    \item 曲名
    \item 作曲者
    \item 作詞者
    \item 編曲者
    \item 歌唱者
    \item URL
\end{itemize}

「曲を追加する」より,追加ページへアクセスする.\\
楽曲名をクリックすることで,各楽曲の詳細表示ページへアクセスする.\\
URLをクリックすることで,各楽曲を再生可能なページへアクセスする.

\section{追加ページ}
\begin{figure}[H]
  \centering
  \includegraphics[width=5cm\textwidth]{fig/os_add_page.png}
  \caption{追加ページ}
  \label{fig:os_add_page}
\end{figure}

追加ページでは,新たに楽曲を追加できる.\\ 
下記の情報を入力できる.
\begin{itemize}
    \item 曲名
    \item 作曲者
    \item 作詞者
    \item 編曲者
    \item 歌唱者
    \item 収録アルバム名
    \item 曲の長さ
    \item URL
    \item 歌詞
\end{itemize}

idは自動的に決定されるので,ユーザーは指定できない.\\
曲名のみ入力必須となっている.\\
送信ボタンを押すと,曲が追加され,一覧表示ページに移動する.

\section{詳細表示ページ}
\begin{figure}[H]
  \centering
  \includegraphics[width=11cm\textwidth]{fig/os_detail_page.png}
  \caption{詳細表示ページ}
  \label{fig:os_detail_page}
\end{figure}

詳細表示ページでは,その楽曲が持つすべての情報を見ることができる.\\
下記の情報が表示される.
\begin{itemize}
    \item id
    \item 曲名
    \item 作曲者
    \item 作詞者
    \item 編曲者
    \item 歌唱者
    \item 収録アルバム名
    \item 曲の長さ
    \item URL
    \item 歌詞
\end{itemize}

URLをクリックすることで,各楽曲を再生可能なページへアクセスする.\\
「一覧に戻る」より,一覧表示ページに戻る.\\
「削除する」より,削除ページへアクセスする.\\
「編集する」より,編集ページへアクセスする.

\section{削除ページ}
\begin{figure}[H]
  \centering
  \includegraphics[width=11cm\textwidth]{fig/os_remove_page.png}
  \caption{削除ページ}
  \label{fig:os_remove_page}
\end{figure}

削除ページでは,曲を削除する際の最終確認が求められる.\\
「キャンセル」より,詳細表示ページに戻る.\\
「削除する」より,曲を削除し,一覧表示ページへアクセスする.

\section{編集ページ}
\begin{figure}[H]
  \centering
  \includegraphics[width=5cm\textwidth]{fig/os_edit_page.png}
  \caption{編集ページ}
  \label{fig:os_edit_page}
\end{figure}

編集ページでは,既存の曲の情報を編集することができる.\\
下記の情報を編集できる.
\begin{itemize}
    \item 曲名
    \item 作曲者
    \item 作詞者
    \item 編曲者
    \item 歌唱者
    \item 収録アルバム名
    \item 曲の長さ
    \item URL
    \item 歌詞
\end{itemize}

idは自動的に決定されるので,ユーザーは編集できない.\\
曲名のみ入力必須となっている.\\
送信ボタンを押すと,曲の情報が編集され,詳細表示ページに移動する.

\section{エラーページ}
\begin{figure}[H]
  \centering
  \includegraphics[width=11cm\textwidth]{fig/error_page.png}
  \caption{エラーページ}
  \label{fig:error_page}
\end{figure}

ページ間を移動する際,曲の情報がうまく取得できないとエラーとなり,このページが表示される.\\
「一覧に戻る」より,一覧表示ページに戻る.